\documentclass[]{article}
\usepackage{lmodern}
\usepackage{amssymb,amsmath}
\usepackage{ifxetex,ifluatex}
\usepackage{fixltx2e} % provides \textsubscript
\ifnum 0\ifxetex 1\fi\ifluatex 1\fi=0 % if pdftex
  \usepackage[T1]{fontenc}
  \usepackage[utf8]{inputenc}
\else % if luatex or xelatex
  \ifxetex
    \usepackage{mathspec}
  \else
    \usepackage{fontspec}
  \fi
  \defaultfontfeatures{Ligatures=TeX,Scale=MatchLowercase}
\fi
% use upquote if available, for straight quotes in verbatim environments
\IfFileExists{upquote.sty}{\usepackage{upquote}}{}
% use microtype if available
\IfFileExists{microtype.sty}{%
\usepackage{microtype}
\UseMicrotypeSet[protrusion]{basicmath} % disable protrusion for tt fonts
}{}
\usepackage[margin=1in]{geometry}
\usepackage{hyperref}
\hypersetup{unicode=true,
            pdftitle={Statistical Inference Course Project},
            pdfauthor={KKher},
            pdfborder={0 0 0},
            breaklinks=true}
\urlstyle{same}  % don't use monospace font for urls
\usepackage{color}
\usepackage{fancyvrb}
\newcommand{\VerbBar}{|}
\newcommand{\VERB}{\Verb[commandchars=\\\{\}]}
\DefineVerbatimEnvironment{Highlighting}{Verbatim}{commandchars=\\\{\}}
% Add ',fontsize=\small' for more characters per line
\usepackage{framed}
\definecolor{shadecolor}{RGB}{248,248,248}
\newenvironment{Shaded}{\begin{snugshade}}{\end{snugshade}}
\newcommand{\AlertTok}[1]{\textcolor[rgb]{0.94,0.16,0.16}{#1}}
\newcommand{\AnnotationTok}[1]{\textcolor[rgb]{0.56,0.35,0.01}{\textbf{\textit{#1}}}}
\newcommand{\AttributeTok}[1]{\textcolor[rgb]{0.77,0.63,0.00}{#1}}
\newcommand{\BaseNTok}[1]{\textcolor[rgb]{0.00,0.00,0.81}{#1}}
\newcommand{\BuiltInTok}[1]{#1}
\newcommand{\CharTok}[1]{\textcolor[rgb]{0.31,0.60,0.02}{#1}}
\newcommand{\CommentTok}[1]{\textcolor[rgb]{0.56,0.35,0.01}{\textit{#1}}}
\newcommand{\CommentVarTok}[1]{\textcolor[rgb]{0.56,0.35,0.01}{\textbf{\textit{#1}}}}
\newcommand{\ConstantTok}[1]{\textcolor[rgb]{0.00,0.00,0.00}{#1}}
\newcommand{\ControlFlowTok}[1]{\textcolor[rgb]{0.13,0.29,0.53}{\textbf{#1}}}
\newcommand{\DataTypeTok}[1]{\textcolor[rgb]{0.13,0.29,0.53}{#1}}
\newcommand{\DecValTok}[1]{\textcolor[rgb]{0.00,0.00,0.81}{#1}}
\newcommand{\DocumentationTok}[1]{\textcolor[rgb]{0.56,0.35,0.01}{\textbf{\textit{#1}}}}
\newcommand{\ErrorTok}[1]{\textcolor[rgb]{0.64,0.00,0.00}{\textbf{#1}}}
\newcommand{\ExtensionTok}[1]{#1}
\newcommand{\FloatTok}[1]{\textcolor[rgb]{0.00,0.00,0.81}{#1}}
\newcommand{\FunctionTok}[1]{\textcolor[rgb]{0.00,0.00,0.00}{#1}}
\newcommand{\ImportTok}[1]{#1}
\newcommand{\InformationTok}[1]{\textcolor[rgb]{0.56,0.35,0.01}{\textbf{\textit{#1}}}}
\newcommand{\KeywordTok}[1]{\textcolor[rgb]{0.13,0.29,0.53}{\textbf{#1}}}
\newcommand{\NormalTok}[1]{#1}
\newcommand{\OperatorTok}[1]{\textcolor[rgb]{0.81,0.36,0.00}{\textbf{#1}}}
\newcommand{\OtherTok}[1]{\textcolor[rgb]{0.56,0.35,0.01}{#1}}
\newcommand{\PreprocessorTok}[1]{\textcolor[rgb]{0.56,0.35,0.01}{\textit{#1}}}
\newcommand{\RegionMarkerTok}[1]{#1}
\newcommand{\SpecialCharTok}[1]{\textcolor[rgb]{0.00,0.00,0.00}{#1}}
\newcommand{\SpecialStringTok}[1]{\textcolor[rgb]{0.31,0.60,0.02}{#1}}
\newcommand{\StringTok}[1]{\textcolor[rgb]{0.31,0.60,0.02}{#1}}
\newcommand{\VariableTok}[1]{\textcolor[rgb]{0.00,0.00,0.00}{#1}}
\newcommand{\VerbatimStringTok}[1]{\textcolor[rgb]{0.31,0.60,0.02}{#1}}
\newcommand{\WarningTok}[1]{\textcolor[rgb]{0.56,0.35,0.01}{\textbf{\textit{#1}}}}
\usepackage{graphicx,grffile}
\makeatletter
\def\maxwidth{\ifdim\Gin@nat@width>\linewidth\linewidth\else\Gin@nat@width\fi}
\def\maxheight{\ifdim\Gin@nat@height>\textheight\textheight\else\Gin@nat@height\fi}
\makeatother
% Scale images if necessary, so that they will not overflow the page
% margins by default, and it is still possible to overwrite the defaults
% using explicit options in \includegraphics[width, height, ...]{}
\setkeys{Gin}{width=\maxwidth,height=\maxheight,keepaspectratio}
\IfFileExists{parskip.sty}{%
\usepackage{parskip}
}{% else
\setlength{\parindent}{0pt}
\setlength{\parskip}{6pt plus 2pt minus 1pt}
}
\setlength{\emergencystretch}{3em}  % prevent overfull lines
\providecommand{\tightlist}{%
  \setlength{\itemsep}{0pt}\setlength{\parskip}{0pt}}
\setcounter{secnumdepth}{0}
% Redefines (sub)paragraphs to behave more like sections
\ifx\paragraph\undefined\else
\let\oldparagraph\paragraph
\renewcommand{\paragraph}[1]{\oldparagraph{#1}\mbox{}}
\fi
\ifx\subparagraph\undefined\else
\let\oldsubparagraph\subparagraph
\renewcommand{\subparagraph}[1]{\oldsubparagraph{#1}\mbox{}}
\fi

%%% Use protect on footnotes to avoid problems with footnotes in titles
\let\rmarkdownfootnote\footnote%
\def\footnote{\protect\rmarkdownfootnote}

%%% Change title format to be more compact
\usepackage{titling}

% Create subtitle command for use in maketitle
\providecommand{\subtitle}[1]{
  \posttitle{
    \begin{center}\large#1\end{center}
    }
}

\setlength{\droptitle}{-2em}

  \title{Statistical Inference Course Project}
    \pretitle{\vspace{\droptitle}\centering\huge}
  \posttitle{\par}
    \author{KKher}
    \preauthor{\centering\large\emph}
  \postauthor{\par}
      \predate{\centering\large\emph}
  \postdate{\par}
    \date{7/18/2020}


\begin{document}
\maketitle

\hypertarget{overview}{%
\subsection{Overview}\label{overview}}

In this project you will investigate the exponential distribution in R
and compare it with the Central Limit Theorem. The second portion of the
project, we're going to analyze the ToothGrowth data in the R datasets
package.

\begin{Shaded}
\begin{Highlighting}[]
\KeywordTok{library}\NormalTok{(ggplot2)}
\KeywordTok{library}\NormalTok{(dplyr)}
\end{Highlighting}
\end{Shaded}

\hypertarget{part-1-simulation-exercise-instructions}{%
\subsection{Part 1: Simulation Exercise
Instructions}\label{part-1-simulation-exercise-instructions}}

The exponential distribution can be simulated in R with rexp(n, lambda)
where lambda is the rate parameter. - Has mean = 1/lambda \& standard
deviation = 1/lambda. - Set lambda = 0.2 - Investigate the distribution
of averages of 40 exponentials. - Note that you will need to do a 1000
simulations.

\begin{Shaded}
\begin{Highlighting}[]
\NormalTok{lambda =}\StringTok{ }\FloatTok{0.2}

\CommentTok{# theoretical Mean & standard deviation}
\NormalTok{actual_mean =}\StringTok{ }\DecValTok{1}\OperatorTok{/}\FloatTok{0.2}
\NormalTok{actual_sd =}\StringTok{ }\DecValTok{1}\OperatorTok{/}\FloatTok{0.2}

\CommentTok{# simulate 1000 }
\NormalTok{sample_means =}\StringTok{ }\OtherTok{NULL}
\ControlFlowTok{for}\NormalTok{ (i }\ControlFlowTok{in} \DecValTok{1} \OperatorTok{:}\StringTok{ }\DecValTok{1000}\NormalTok{) sample_means =}\StringTok{ }\KeywordTok{c}\NormalTok{(sample_means, }\KeywordTok{mean}\NormalTok{(}\KeywordTok{rexp}\NormalTok{(}\DecValTok{40}\NormalTok{, lambda)))}
\KeywordTok{ggplot}\NormalTok{() }\OperatorTok{+}
\StringTok{  }\KeywordTok{aes}\NormalTok{(sample_means) }\OperatorTok{+}
\StringTok{  }\KeywordTok{geom_histogram}\NormalTok{(}\DataTypeTok{fill=}\StringTok{"#69b3a2"}\NormalTok{, }\DataTypeTok{color=}\StringTok{"#e9ecef"}\NormalTok{, }\DataTypeTok{alpha=}\FloatTok{0.9}\NormalTok{) }\OperatorTok{+}\StringTok{ }
\StringTok{  }\KeywordTok{ggtitle}\NormalTok{(}\StringTok{"Distribution of 1000 simulations of averages of 40 exponentials"}\NormalTok{) }\OperatorTok{+}
\StringTok{  }\KeywordTok{geom_vline}\NormalTok{(}\DataTypeTok{xintercept =} \KeywordTok{mean}\NormalTok{(sample_means), }\DataTypeTok{color=}\StringTok{"blue"}\NormalTok{)}
\end{Highlighting}
\end{Shaded}

\includegraphics{R_markdown_code_files/figure-latex/Exponential Simulation-1.pdf}

Actual Mean = \texttt{r\ actual\_mean} and the simulated mean =
\texttt{r\ mean(sample\_means)}

Actual sd = \texttt{r\ actual\_sd} and the simulated mean =
\texttt{r\ sd(sample\_means)}

\hypertarget{part-2-basic-inferential-data-analysis-instructions}{%
\subsection{Part 2: Basic Inferential Data Analysis
Instructions}\label{part-2-basic-inferential-data-analysis-instructions}}

\begin{itemize}
\tightlist
\item
  Load the ToothGrowth data and perform some basic exploratory data
  analyses
\end{itemize}

\begin{Shaded}
\begin{Highlighting}[]
\KeywordTok{data}\NormalTok{(ToothGrowth)}
\end{Highlighting}
\end{Shaded}

\begin{itemize}
\tightlist
\item
  Provide a basic summary of the data.
\end{itemize}

\begin{Shaded}
\begin{Highlighting}[]
\KeywordTok{head}\NormalTok{(ToothGrowth)}
\end{Highlighting}
\end{Shaded}

\begin{verbatim}
##    len supp dose
## 1  4.2   VC  0.5
## 2 11.5   VC  0.5
## 3  7.3   VC  0.5
## 4  5.8   VC  0.5
## 5  6.4   VC  0.5
## 6 10.0   VC  0.5
\end{verbatim}

\begin{Shaded}
\begin{Highlighting}[]
\KeywordTok{summary}\NormalTok{(ToothGrowth)}
\end{Highlighting}
\end{Shaded}

\begin{verbatim}
##       len        supp         dose      
##  Min.   : 4.20   OJ:30   Min.   :0.500  
##  1st Qu.:13.07   VC:30   1st Qu.:0.500  
##  Median :19.25           Median :1.000  
##  Mean   :18.81           Mean   :1.167  
##  3rd Qu.:25.27           3rd Qu.:2.000  
##  Max.   :33.90           Max.   :2.000
\end{verbatim}

\begin{Shaded}
\begin{Highlighting}[]
\KeywordTok{str}\NormalTok{(ToothGrowth)}
\end{Highlighting}
\end{Shaded}

\begin{verbatim}
## 'data.frame':    60 obs. of  3 variables:
##  $ len : num  4.2 11.5 7.3 5.8 6.4 10 11.2 11.2 5.2 7 ...
##  $ supp: Factor w/ 2 levels "OJ","VC": 2 2 2 2 2 2 2 2 2 2 ...
##  $ dose: num  0.5 0.5 0.5 0.5 0.5 0.5 0.5 0.5 0.5 0.5 ...
\end{verbatim}

\begin{Shaded}
\begin{Highlighting}[]
\CommentTok{# plot len & supp}
\KeywordTok{ggplot}\NormalTok{(}\DataTypeTok{data=}\NormalTok{ToothGrowth) }\OperatorTok{+}\StringTok{ }
\StringTok{  }\KeywordTok{aes}\NormalTok{(}\DataTypeTok{x=}\NormalTok{supp, }\DataTypeTok{y=}\NormalTok{len)}\OperatorTok{+}
\StringTok{  }\KeywordTok{geom_boxplot}\NormalTok{(}\KeywordTok{aes}\NormalTok{(}\DataTypeTok{fill=}\NormalTok{supp)) }\OperatorTok{+}
\StringTok{  }\KeywordTok{facet_grid}\NormalTok{(}\DataTypeTok{cols =} \KeywordTok{vars}\NormalTok{(dose)) }\OperatorTok{+}
\StringTok{  }\KeywordTok{ggtitle}\NormalTok{(}\StringTok{"Length-Supplement Relation split by dose"}\NormalTok{)}
\end{Highlighting}
\end{Shaded}

\includegraphics{R_markdown_code_files/figure-latex/toothGrowth_EDA-1.pdf}

From Figure above, mean seems to be equal for both supp only for dose=2.

\begin{itemize}
\tightlist
\item
  Use confidence intervals and/or hypothesis tests to compare tooth
  growth by supp and dose. (Only use the techniques from class, even if
  there's other approaches worth considering)
\end{itemize}

\begin{Shaded}
\begin{Highlighting}[]
\CommentTok{# create t test}

\CommentTok{# perform t test between supp types where dose = 2}
\KeywordTok{t.test}\NormalTok{(ToothGrowth}\OperatorTok{$}\NormalTok{len[ToothGrowth}\OperatorTok{$}\NormalTok{supp}\OperatorTok{==}\StringTok{"OJ"} \OperatorTok{&}\StringTok{ }\NormalTok{ToothGrowth}\OperatorTok{$}\NormalTok{dose}\OperatorTok{==}\DecValTok{2}\NormalTok{], ToothGrowth}\OperatorTok{$}\NormalTok{len[ToothGrowth}\OperatorTok{$}\NormalTok{supp}\OperatorTok{==}\StringTok{"VC"} \OperatorTok{&}\StringTok{ }\NormalTok{ToothGrowth}\OperatorTok{$}\NormalTok{dose}\OperatorTok{==}\DecValTok{2}\NormalTok{], }\DataTypeTok{paired =} \OtherTok{TRUE}\NormalTok{)}
\end{Highlighting}
\end{Shaded}

\begin{verbatim}
## 
##  Paired t-test
## 
## data:  ToothGrowth$len[ToothGrowth$supp == "OJ" & ToothGrowth$dose ==  and ToothGrowth$len[ToothGrowth$supp == "VC" & ToothGrowth$dose ==     2] and     2]
## t = -0.042592, df = 9, p-value = 0.967
## alternative hypothesis: true difference in means is not equal to 0
## 95 percent confidence interval:
##  -4.328976  4.168976
## sample estimates:
## mean of the differences 
##                   -0.08
\end{verbatim}

\begin{Shaded}
\begin{Highlighting}[]
\CommentTok{# perform t test between supp types where dose != 2}
\KeywordTok{t.test}\NormalTok{(ToothGrowth}\OperatorTok{$}\NormalTok{len[ToothGrowth}\OperatorTok{$}\NormalTok{supp}\OperatorTok{==}\StringTok{"OJ"} \OperatorTok{&}\StringTok{ }\NormalTok{ToothGrowth}\OperatorTok{$}\NormalTok{dose}\OperatorTok{!=}\DecValTok{2}\NormalTok{], ToothGrowth}\OperatorTok{$}\NormalTok{len[ToothGrowth}\OperatorTok{$}\NormalTok{supp}\OperatorTok{==}\StringTok{"VC"} \OperatorTok{&}\StringTok{ }\NormalTok{ToothGrowth}\OperatorTok{$}\NormalTok{dose}\OperatorTok{!=}\DecValTok{2}\NormalTok{], }\DataTypeTok{paired =} \OtherTok{TRUE}\NormalTok{)}
\end{Highlighting}
\end{Shaded}

\begin{verbatim}
## 
##  Paired t-test
## 
## data:  ToothGrowth$len[ToothGrowth$supp == "OJ" & ToothGrowth$dose !=  and ToothGrowth$len[ToothGrowth$supp == "VC" & ToothGrowth$dose !=     2] and     2]
## t = 4.6042, df = 19, p-value = 0.0001936
## alternative hypothesis: true difference in means is not equal to 0
## 95 percent confidence interval:
##  3.048852 8.131148
## sample estimates:
## mean of the differences 
##                    5.59
\end{verbatim}

\begin{itemize}
\tightlist
\item
  State your conclusions and the assumptions needed for your
  conclusions.
\end{itemize}

If we consider Null Hypothesis (H0) to be mean for supp is the same for
each dose

1- We Fail to reject H0 where dose = 2, as t-value is very small and
equal to
\texttt{r\ t.test(ToothGrowth\$len{[}ToothGrowth\$supp=="OJ"\ \&\ ToothGrowth\$dose==2{]},\ ToothGrowth\$len{[}ToothGrowth\$supp=="VC"\ \&\ ToothGrowth\$dose==2{]},\ paired\ =\ TRUE)\$statistic}
1- We reject H0 where dose does not equal to 2, as t-value is large
enough and equal to
\texttt{r\ t.test(ToothGrowth\$len{[}ToothGrowth\$supp=="OJ"\ \&\ ToothGrowth\$dose!=2{]},\ ToothGrowth\$len{[}ToothGrowth\$supp=="VC"\ \&\ ToothGrowth\$dose!=2{]},\ paired\ =\ TRUE)\$statistic}


\end{document}
